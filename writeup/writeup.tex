\documentclass[11pt]{article}
\usepackage[letterpaper, margin=0.8in]{geometry}
\usepackage{amsmath}
\usepackage{amssymb}
\usepackage{wrapfig}
\usepackage[makeroom]{cancel}
\usepackage{bbm}
\usepackage{booktabs}
\usepackage{float}
\usepackage{array}
\usepackage{bm}
\usepackage{enumerate}
\usepackage{amsfonts}
\usepackage{color}
\usepackage{hyperref}
\usepackage{xcolor}
\usepackage{hyperref}
\usepackage{newfloat}
\usepackage{graphicx}
\usepackage{caption}
\usepackage{bbm}

\begin{document}
\section*{Goal}
Does local ancestry affect the gene network?

\section*{Two nodes and their correlations}
Consider two genes $k_1$, $k_2$ and the corresponding gene expression vector of $y_{k_1}$ and $y_{k_2}$, each of length $n$ where $n$ is the number of subjects. We also have two local ancestry vectors $x_{k_1}$ and $x_{k_2}$. The temporary question of interest is whether the correlation between $y_{k_1}$ and $y_{k_2}$ is driven by either $x_{k_1}$ or $x_{k_2}$.\\

\noindent
Without loss of generality, let's see how much the correlation between the two expression levels changes with respect to the values of $x_{k_1}$. Each element of the local ancestry vector $x$ takes one of three values : 0, 1, or 2. When we divide the $n$ samples into three groups based on local ancestry each with size $n_0$, $n_1$, $n_2$, we can observe three correlation coefficients: $\rho_0$, $\rho_1$, and $\rho_2$. Using Fisher's transformation, we consider three normally distributed $z$ values for each group : $z_0$, $z_1$, and $z_2$. 
$$z_i = \frac{1}{2} ln \left( \frac{1+\rho_i}{1-\rho_i}\right) \sim N\left(\frac{1}{2} ln\left(\frac{1+\tilde{\rho_i}}{1-\tilde{\rho}_i}\right), \frac{1}{\sqrt{n_i-3}}\right)$$
\noindent where $\tilde{\rho}_i$ is the underlying true correlation coefficient for the group with local ancestry $i$. \\

\noindent We assume that $z_0$, $z_1$, and $z_2$ have linear relationship. We formulate the model like below.
$$\begin{bmatrix} z_0 \\ z_1 \\ z_2 \end{bmatrix}
\sim MVN \left(
\begin{bmatrix} \mu \\ \mu+\beta \\ \mu+2\beta \end{bmatrix},
\begin{bmatrix}
\frac{1}{n_0-3} & 0 & 0\\
0 & \frac{1}{n_1-3} & 0\\
0 & 0 & \frac{1}{n_2-3}
\end{bmatrix} \right)$$
with hypotheses
$$H_0: \beta = 0$$
$$H_A: \beta \neq 0$$
\noindent For notational convenience, we define the following matrix and vectors.
$$\Sigma = \begin{bmatrix}
\frac{1}{n_0-3} & 0 & 0\\
0 & \frac{1}{n_1-3} & 0\\
0 & 0 & \frac{1}{n_2-3}
\end{bmatrix} = \begin{bmatrix}
\sigma_0^2 & 0 & 0\\
0 & \sigma_1^2 & 0\\
0 & 0 & \sigma_2^2
\end{bmatrix} $$
$$\bm{\mu} = \begin{bmatrix} \mu \\ \mu \\ \mu \end{bmatrix}, \hspace{5mm} \bm{\mu_{\beta}} = \begin{bmatrix} \mu \\ \mu + \beta \\ \mu+2\beta \end{bmatrix}, \hspace{5mm} \bm{z} = \begin{bmatrix} z_0 \\ z_1 \\ z_2 \end{bmatrix}$$
\noindent Also for notational convenience, we introduce $\Phi$ and $\Lambda$. 
$$ \Phi = \sigma_0^2 \sigma_1^2 + \sigma_1^2 \sigma_2^2 + \sigma_2^2 \sigma_0^2$$
$$\Lambda = \sigma_1^2 \sigma_2^2 z_0 + \sigma_0^2 \sigma_2^2 z_1 +\sigma_0^2 \sigma_1^2 z_2$$
\noindent Now, we set up the log likelihood of the two cases: when $\beta$ is 0 and when $\beta$ is the MLE estimator.
\begin{align}
\ell_{\beta = 0} &= -\frac{1}{2} log |2\pi \Sigma| -\frac{1}{2} ((\bm{z}-\bm{\mu})^T \Sigma^{-1}(\bm{z}-\bm{\mu}))\\
\frac{\partial \ell_{\beta = 0}}{\partial \mu} &= \frac{z_0 -\mu}{\sigma_0^2} + \frac{z_1-\mu}{\sigma_1^2} + \frac{z_2-\mu}{\sigma_2^2} = 0\\
&\Rightarrow \hat{\mu}_{0,MLE} = \frac{\sigma_1^2 \sigma_2^2 z_0 + 
\sigma_2^2 \sigma_0^2 z_1 + \sigma_0^2 \sigma_1^2 z_2
}{\sigma_0^2 \sigma_1^2 + \sigma_1^2 \sigma_2^2 + \sigma_2^2 + \sigma_0^2} = \frac{\Lambda}{\Phi}
\end{align}
\noindent is the MLE estimator for the $\mu$ under the null hypothesis. \\

\noindent For the alternative hypothesis, 
\begin{align*}
\ell_{\beta \neq 0}&= -\frac{1}{2} log|2\pi\Sigma| - \frac{1}{2} ((\bm{z}-\bm{\mu_{\beta}})^T \Sigma^{-1} (\bm{z}-\bm{\mu_{\beta}}))\\
\frac{\partial \ell_{\beta \neq 0}}{\partial \mu} &=
\frac{z_0 - \mu}{\sigma_0^2}+ \frac{z_1 - \mu - \beta}{\sigma_1^2} + \frac{z_2-\mu-2\beta}{\sigma_2^2}\\
&\Rightarrow \hat{\mu}_{A,MLE} = \frac{\sigma_1^2 \sigma_2^2 z_0 + 
\sigma_2^2 \sigma_0^2 (z_1-\beta) + \sigma_0^2 \sigma_1^2 (z_2-2\beta)
}{\sigma_0^2 \sigma_1^2 + \sigma_1^2 \sigma_2^2 + \sigma_2^2 \sigma_0^2} = \frac{\Lambda - \sigma_2^2 \sigma_0^2 \beta - 2\sigma_0^2 \sigma_1^2 \beta}{\Phi}
\end{align*}
\noindent is the MLE estimator for $\mu$ under the alternative hypothesis. Now we plug in the above to get the MLE estimator for $\beta$. 
\begin{align*}
\frac{\partial \ell_{\beta \neq 0}}{\partial \beta} &= \frac{z_1-\mu-\beta}{\sigma_1^2} + \frac{2(z_2-\mu-2\beta)}{\sigma_2^2}\\
&\Rightarrow  \sigma_2^2(z_1-\hat{\mu}_{A,MLE}-\hat{\beta}_{A,MLE}) + 2\sigma_1^2 (z_2-\hat{\mu}_{A,MLE}-2\hat{\beta}_{A,MLE}) = 0\\
&\Rightarrow \frac{
\Phi z_1 - \Lambda + (\sigma_0^2 \sigma_2^2 + 2\sigma_0^2 \sigma_1^2 -\Phi) \hat{\beta}_{A,MLE}
}{\Phi \sigma_1^2} + \frac{
2(
\Phi z_2 - \Lambda + (\sigma_0^2 \sigma_2^2 + 2\sigma_0^2 \sigma_1^2 - 2\Phi) \hat{\beta}_{A,MLE}
)}{\Phi \sigma_2^2} = 0\\
&\Rightarrow \hat{\beta}_{A,MLE} = \frac{
(\sigma_2^2 + 2\sigma_1^2)\Lambda - (\sigma_2^2 z_1 + 2\sigma_1^2 z_2) \Phi
}{
(\sigma_2^2 + 2\sigma_1^2) (\sigma_0^2 \sigma_2^2 + 2\sigma_0^2 \sigma_1^2 - 2\Phi)
}
\end{align*}

\noindent Now we can perform the likelihood ratio test and see if $\beta$ is significantly different from $0$. Under the null,
$$-2log \left(
\frac{N_3(\bm{z}; \bm{\hat{\mu}}, \Sigma)}{N_3(\bm{z}; \bm{\hat{\mu}_{\beta}}, \Sigma)}
\right)\sim \chi_1^2$$
and we can compute the corresponding $p$-value.
\end{document}